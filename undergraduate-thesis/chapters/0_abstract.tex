%%
% BIThesis 本科毕业设计论文模板 —— 使用 XeLaTeX 编译 The BIThesis Template for Undergraduate Thesis
% This file has no copyright assigned and is placed in the Public Domain.
%%

% 中英文摘要章节
\begin{abstract}
针对图持续学习中因数据动态演化和复杂结构依赖导致的灾难性遗忘问题,本文提出了一种基于离散数学核心工具——图编辑距离(GED)的动态建模新范式。该方法利用GED的原子编辑操作量化图结构演化模式,将宏观序列差异映射为顶点活跃度与子结构持久性等微观拓扑语义标签。在此基础上,本文构建了分层动态建模框架并提出两种创新策略:基于GED的经验回放通过优先保留高持久性核心子图以巩固长期记忆;拓扑感知正则化则将顶点活跃度融入参数重要性估计,对稳定结构对应的模型参数施加更强保护。此外,为解决GED计算的NP-hard难题,设计了高效近似算法(BGM-GED)及增量优化策略。本研究通过融合图论原理与机器学习需求,构建了具有强可解释性的图持续学习框架,有效平衡了模型的“记忆”与“适应”能力,为动态图学习提供了新的理论依据与方法路径。

\end{abstract}

% 英文摘要章节
\begin{abstractEn}
To address the challenge of catastrophic forgetting in Graph Continual Learning (GCL) caused by dynamic evolution and complex structural dependencies, this thesis proposes a novel dynamic modeling paradigm based on Graph Edit Distance (GED). By quantifying evolutionary patterns through atomic edit operations, the proposed method maps macroscopic sequence differences into microscopic topological semantic labels, such as vertex activeness and substructure persistence. Based on this, a hierarchical framework is established with two novel strategies: GED-Guided Replay, which consolidates long-term memory by prioritizing high-persistence core subgraphs, and Topology-Aware Regularization, which incorporates vertex activeness into parameter importance estimation to impose stronger constraints on stable structures. Furthermore, an efficient approximation algorithm (BGM-GED) and incremental optimization strategies are designed to overcome the NP-hard computational complexity. By integrating graph theory principles with machine learning demands, this research constructs a highly interpretable GCL framework that effectively balances stability and plasticity, providing a robust theoretical foundation and methodological path for dynamic graph learning.

\end{abstractEn}
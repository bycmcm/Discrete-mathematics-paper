%%
% BIThesis 本科毕业设计论文模板 —— 使用 XeLaTeX 编译 The BIThesis Template for Undergraduate Thesis
% This file has no copyright assigned and is placed in the Public Domain.
%%

% 第一章节
\chapter{绪论}

\section{研究背景与意义}
随着社交网络、知识图谱、生物信息网络等复杂系统的蓬勃发展,图结构数据已成为刻画实体间关联关系的重要载体。在这些现实应用中,图数据并非静态不变,而是随着时间不断演化——新的节点与边持续加入,旧的关系可能减弱或消失,图拓扑与属性处于动态变化之中。如何让机器学习模型在这种动态图序列中持续、高效地学习,并能在适应新知识的同时牢固记忆历史模式,已成为当前人工智能领域的前沿挑战,这一研究方向被称为图持续学习。

\section{现有挑战}

图持续学习的核心难点在于灾难性遗忘:当模型基于梯度下降优化学习新到达的图快照时,其参数更新往往会损害对先前图结构的表示能力。然而,与图像或文本的持续学习不同,图数据中样本(节点/边)间存在复杂的结构依赖,其演化蕴含丰富的拓扑语义,这要求遗忘缓解机制必须与图的结构特性深度结合。

离散数学,尤其是图论,为理解和建模这种结构演化提供了坚实的理论基础。其中,图编辑距离(GED)作为一种经典的图相似性度量,能够将两个图之间的差异转化为一系列可解释的原子操作(如顶点/边的插入、删除、替换),并量化其最小转换代价。GED不仅度量了变化的幅度,其蕴含的编辑路径更揭示了图结构演化的具体模式。这为我们提供了一种强有力的数学工具:将图持续学习中“记忆与适应”的平衡问题,转化为对图结构演化轨迹的量化分析与针对性干预。

因此,本研究旨在系统性地探索离散数学(特别是图编辑距离)在图持续学习中的应用。我们提出一个基于GED的动态建模框架,旨在从动态图序列中提取具有明确拓扑语义的演化特征,并据此设计新颖的持续学习策略。本研究不仅深化了图论在机器学习中的应用,也为构建更稳健、更高效的动态图学习模型提供了新的思路与方法。

\chapter{离散数学基础}

\section{图的基本定义}

在离散数学中,图是描述事物间二元关系的一种抽象数学模型,
一个图 $G$ 可以定义为一个有序二元组 $G=(V,E)$,其中V 是一个有限的、非空的集合,称为顶点集,$E$是顶点集 $V$ 中无序对的集合
(对于无向图)或有序对的集合(对于有向图),称为边集。其核心在于忽略事物的具体属性,仅关注对象之间是否存在关联。

相比之下,图持续学习关注的是模型如何从一个按时间顺序到达的图序列中持续地、高效地学习并积累知识,
同时避免对已学旧知识的灾难性遗忘。在此语境下,基础的分析单元不再是单个静态图,而是一个动态演化的图序列。

\section{子图与同构}

在对动态图序列的分析中,我们不仅需要描述单个图的结构,更需要精确地刻画不同图之间的结构关系。子图关系描述了图之间的包含与嵌入,而同构关系则定义了图之间结构上的完全等价性。这两个概念是理解图编辑距离与图演化模式的理论基础。

\subsection{子图}

子图关系反映了图结构之间的部分与整体,是分析局部模式、识别功能模块以及量化结构变化的核心工具。

在离散数学的定义中,令$G=(V,E)$和$G'=(V',E')$为两个图,若$V'\subseteq V$且$E'\subseteq E$,并且$E'$中的每条边所关联的顶点均属于$V'$,称$G'$是$G$的子图。
而子图在图持续学习中有着重要意义,其核心动态图序列的演化可以自然地看成一个子图扩展过程,即$G_t$可视为由子图$G_(t-1)$演化而来,新增部分则代表了在时间间隔内发生的变化。这种视角将全局演化问题转化为局部结构增量的识别问题。

\subsection{图同构:结构的等价性}\
当我们需要判断两个图在结构上是否“完全相同”时,就需要用到离散数学中图同构的概念。同构是图论中最深刻的等价关系,也是许多图算法(如图匹配、图数据库查询)的核心。
两个图$G=(V,E)$和$H=(U,F)$是同构的,即$G\cong H$,当且仅当存在一个双射函数$\phi:V→U$,使得
对V中任意两个顶点$u,v$,
\begin{align}
[u,v]\in E \iff [\varphi(u),\varphi(v)]\in F.
\end{align}
这样的函数 $\varphi$ 称为从 $G$ 到 $H$ 的一个同构映射。对于有向图何标记图,则条件要分别加强到有向边相同和标签相同。

同构关系表明,两个图虽然在表面表示上可能不同(顶点编号、空间布局),但其连接拓扑结构完全一致。
子图同构判定是NP完全问题,它在多个领域有根本性应用:在化学信息学中,用于查找分子数据库中包含特定功能基团的分子;在社交网络分析中,用于检测网络中的特定社群结构(如小团体、星型结构);在计算机视觉中,用于从场景图中识别已知的物体关系模式。

\subsection{子图同构与图匹配问题}

一个更普遍且在实践中至关重要的难题是子图同构问题,它询问一个图是否包含另一个图的同构副本。

在图持续学习的动态建模背景下,子图同构的概念为我们提供了一种定性地描述结构稳定性的工具。然而,当 H 在演化过程中发生微小畸变(如一名成员暂时断开连接)时,严格的子图同构关系将不再成立,但结构相似性依然很高。此时,我们需要一种定量的、能够容忍微小差异的度量,来刻画这种近似包含或近似等价的关系——这正是接下来的图编辑距离所要解决的问题。

\section{图编辑距离 (Graph Edit Distance, GED)}

在动态图序列的分析中,我们不仅需要识别结构的定性与定量变化,更需要一种能够精确量化这种变化、并能揭示变化细节的数学工具。图编辑距离(Graph Edit Distance, GED)正是这样一种工具,它将图之间的差异转化为一系列可解释的原子操作,并通过最小化操作总代价来定义距离。

\subsection{图编辑距离的形式化定义}

对于两个标记图$G_1=(V_1,E_1,l_(v_1),l_(E_1))$和$G_2=(V_2,E_2,l_(V_2),l_(E_2))$,基本的图编辑操作包括:
\begin{enumerate}[label=\arabic*.]
  \item 顶点替换:将图 $G_1$ 中的一个顶点 $u\in V_1$ 映射为图 $G_2$ 中的一个顶点 $U\in V_2$,记为
        $(u\rightarrow U)$。若 $l_V(u)\neq l_V(U)$,则此操作实质为修改顶点标签。

  \item 顶点插入:将图 $G_2$ 中的一个顶点 $U\in V_2$ 映射到一个特殊的空顶点 $\varepsilon$,记为
        $(\varepsilon\rightarrow U)$,表示在 $G_1$ 中插入该顶点。

  \item 顶点删除:将图 $G_1$ 中的一个顶点 $u\in V_1$ 映射到空顶点 $\varepsilon$,记为
        $(u\rightarrow\varepsilon)$,表示从 $G_1$ 中删除该顶点。

  \item 边替换、边插入、边删除:定义与顶点操作类似,作用于边集。
\end{enumerate}

一个编辑路径 $P=(o_1,o_2,\dots,o_k)$ 是一个有限的编辑操作序列,这些操作将图 $G_1$ 逐步转换为一个与 $G_2$ 同构的图。

基于此,我们可以给定一个代价函数$c$,为每一个编辑操作$o$赋予一个非负代价$c(o) \geq 0$。图$G_1$和$G_2$间的
图编辑距离定义为所有可能编辑路径中的最小总代价:
\begin{align}
\mathrm{GED}(G_1,G_2)
= \min_{P\in\mathcal{P}(G_1,G_2)} \sum_{o\in P} c(o),
\end{align}
其中 $\mathcal{P}(G_1,G_2)$ 是所有能将 $G_1$ 转换为 $G_2$ 的编辑路径的集合。

\subsection{计算挑战与算法概览}
图编辑距离的计算是 \textsc{NP}-hard 问题,且对于一般的图,其判定问题
(给定两个图和阈值 $K$,判断 $\mathrm{GED}$ 是否小于等于 $K$)是 \textsc{NP}-完全的。

这一复杂性源于编辑路径数量的组合爆炸。给定两个具有 $n$ 和 $m$ 个顶点的图,可能的顶点映射
(即顶点替换的集合)数量高达 $O(n^m)$ 量级。因此,在实际应用中,尤其是大规模图或动态图序列中
需要频繁计算时,必须依赖近似算法或启发式方法。以下是三类主要方法:

\begin{enumerate}[label=\arabic*.]
  \item \textbf{基于搜索的精确算法:} 例如 A$^*$ 算法,它利用启发式函数剪枝搜索空间,在中小规模图上
        可求得精确解,但最坏情况下仍是指数复杂度。

  \item \textbf{整数线性规划(ILP)模型:} 将 $\mathrm{GED}$ 计算建模为一个整数线性规划问题,利用优化求解器
        (如 Gurobi、CPLEX)求解。该方法能获得精确解,但同样受限于问题规模。

  \item \textbf{近似与启发式算法:} 这是处理大规模图的主要途径。
        \begin{itemize}
          \item \textbf{二分图匹配法:} 将顶点映射问题转化为二分图最小权匹配问题,通过忽略部分边结构
                获得多项式时间的近似解。著名的 Hungarian 算法和 Volgenant-Jonker 算法常被用于此。
          \item \textbf{局部搜索与迭代改进:} 如 Beam Search(束搜索),在搜索过程中只保留有限数量的
                最有希望的路径。
          \item \textbf{基于深度学习的 $\mathrm{GED}$ 预测:} 近年来,利用图神经网络(GNN)直接学习图的表示
                并预测 $\mathrm{GED}$ 值的方法展现出潜力,能够实现快速但近似的距离估计。
        \end{itemize}
\end{enumerate}
\chapter{图持续学习基础}
我们在此处通过系统性地阐述图持续学习的基本范式,以明晰其核心概念、独特挑战以及与其它学习范式的区别,并分类阐述其典型应用场景,从而为后续“基于图编辑距离的动态建模”提供明确的问题定义和需求背景。
\section{持续学习的核心概念}

持续学习旨在模拟智能体在动态环境中持续积累并整合知识的能力,其核心是处理数据分布的非平稳性。

作为一种机器学习范式,持续学习要求模型从一系列按时间或顺序到达的任务或数据分布
$D_1,D_2,\dots,D_T$ 中依次进行学习。其核心目标是,在学习新任务 $D_t$ 的同时,
能够最大限度地保持对先前所有已学任务 $\{D_1,\dots,D_{t-1}\}$ 的性能。

灾难性遗忘是持续学习中最根本的挑战。当神经网络等模型使用基于梯度的优化方法学习新任务时,
其参数会为了最小化新任务的损失而更新,这一过程往往会覆盖或破坏存储在过去参数中
关于旧任务的知识,导致模型在旧任务上的性能急剧下降,仿佛“遗忘”了之前所学。

为了缓解灾难性遗忘,图持续学习主要围绕三大策略展开研究,分别是:基于回放的方法,基于正则化的方法,基于动态架构的方法。
而当持续学习的对象从独立的图像或文本样本,转变为内部存在复杂结构关联的图数据时,上述挑战和策略被置于一个全新的、更复杂的维度中。
\section{图数据的持续学习场景}

图持续学习并非简单地将持续学习技术应用于图神经网络。图数据自身的结构性、关联性和演化特性,使得其持续学习问题呈现出独特的场景和更高的复杂度。

图数据因为其自身的结构关联,演化特性,在持续学习问题上有更独特更复杂的场景,无论是在节点粒度预测时的分类、回归任务,还是在动态演化的图数据中进行适应、学习,
都对算法的计算效率提出了极高的要求。

为得到最优的算法,我们必须借助图编辑距离,以此量化一个图结构的变化程度,据此解析出具体的演化过程,帮助机器识别出所需关注的子图。

\chapter{基于GED的图动态建模}

基于离散数学和图编辑距离,我们提出一个分层框架:首先,利用GED构建动态演化的量化模型,将宏观序列差异映射为微观结构单元的演化属性;其次,基于此量化模型,设计两种新颖的、具有明确拓扑语义的持续学习策略——基于GED的经验回放与基于GED的拓扑正则化。该框架使得模型不仅能感知变化的大小,更能理解变化的性质,从而实现更精准的知识巩固与适应。

\section{动态演化的量化模型}
给定动态图序列,我们构造一个加权的有向演化图 $\mathcal{G}=(\mathcal{V},\mathcal{A},w)$,其中:
\begin{itemize}[leftmargin=*,nosep]
  \item 顶点集 $\mathcal{V}$:每个顶点 $V_t$ 对应原序列中的一个图快照 $G_t$。
  \item 弧集 $\mathcal{A}$:对于任意 $t\ge 2$,存在一条从 $V_{t-1}$ 指向 $V_t$ 的弧 $a_{t-1,t}$。
  \item 权重函数 $w$:弧 $a_{t-1,t}$ 的权重 $w_{t-1,t}=\mathrm{GED}(G_{t-1},G_t)$。
\end{itemize}
演化图 $\mathcal{G}$ 是动态序列的一阶宏观抽象。通过分析其权重序列 $\{w_{t-1,t}\}$,我们可以识别系统演化的模式:
\begin{itemize}[leftmargin=*,nosep]
  \item \textbf{平稳演化期:} 连续出现较小且波动不大的 GED 值。
  \item \textbf{剧变期:} GED 值出现显著峰值,指示重大结构重组或外部事件冲击。
  \item \textbf{趋势分析:} 计算 GED 序列的移动平均与自相关系数,可判断演化是趋向稳定还是持续动荡。
\end{itemize}
为了将宏观演化与微观结构变化联系起来,我们引入顶点演化轨迹与结构单元活跃度的概念。

对于一个在时间窗 $[t_s,t_e]$ 内存在的顶点 $u$,其演化轨迹 $\mathrm{Traj}(u)$ 是一个序列,记录其在每个快照
$G_t\;(t_s\le t\le t_e)$ 中的邻接向量(或高阶表征)及标签(如有)。顶点的活跃度 $\alpha(u)$ 定义为在其生命周期内,
其邻接向量(或表征)的逐时间步变化量的范数之和,该变化可通过对比相邻快照中的局部子图的 GED 来近似计算。
高 $\alpha(u)$ 表明该顶点连接关系不稳定,是网络中的活跃点或边缘点。

对于一条边 $e=(u,v)$,其持久性 $p(e)$ 定义为在连续快照中同时包含顶点 $u$ 和 $v$ 且边 $e$ 存在的时间步比例。
对于一个连通子图 $H$(如 $k$-clique、$k$-core),其持久性 $p(H)$ 定义为在连续快照中能找到与 $H$ 同构
(或近似同构,GED 小于阈值 $\tau$)的子图的时间步比例。

高持久性的边或子图是网络的骨干结构或功能模块,如长期稳定的合作团队、频繁交互的蛋白复合物。
它们是需要被模型长期牢固记忆的知识实体。

\textbf{基于 GED 的量化流程:}
\begin{enumerate}[label=\arabic*.]
  \item 输入:动态图序列 $\mathbb{G}=\{G_1,G_2,\dots,G_T\}$。
  \item 解析编辑路径:从 $P_{t-1,t}$ 中提取涉及的顶点和边操作,更新对应顶点和边的活跃度与持久性计数器。
  \item 输出量化指标:每个顶点的活跃度 $\alpha(u)$、每条边的持久性 $p(e)$、每个显著子图模式 $H$ 的持久性 $p(H)$,
        以及演化图。
\end{enumerate}
此量化模型将GED的数值输出转化为一系列附着于图结构元素上的、具有明确拓扑语义的属性标签,为后续的持续学习策略提供了精准的方向标。
\section{动态建模策略}
基于上述量化模型,我们提出两种核心的持续学习策略。它们共享一个核心理念:依据结构单元在演化中的角色,实施差异化的学习处理。
\subsection{策略一:基于GED的经验回放}

传统经验回放通常随机或基于多样性选择节点/边。我们提出 GED-Guided Replay (GGR),其核心是利用量化模型输出的持久性指标,智能选择回放样本,确保模型持续复习网络中最稳定、最核心的知识。
GGR的核心步骤为:
\begin{enumerate}[label=\arabic*.]
  \item 识别候选记忆单元:在历史快照中,识别所有满足 $p(e)>\theta_p$(高持久性边)的边,
        以及由这些边诱导出的连通子图作为核心子图。同时,将活跃度极低 $\alpha(u)<\theta_\alpha$ 的顶点标记为稳定顶点。

  \item 构建记忆缓冲区:记忆缓冲区 $\mathcal{M}$ 存储两种类型的记忆项:
        \begin{itemize}[leftmargin=*,nosep]
          \item 子图记忆项:对于每个核心子图 $H$,存储其拓扑结构 $A_H$、顶点特征 $X_H$ 以及在
                上次被观测到时模型对其内部节点的预测输出(软标签)。
          \item 节点上下文记忆项:对于每个稳定顶点 $u$,存储其当前特征 $c_u$、其 $k$-hop 邻域的子图结构、
                以及模型对其的预测输出。
        \end{itemize}

  \item 自适应回放调度:当在新快照 $G_t$ 上训练时,设计回放概率。对于记忆项 $m_i$,其被抽中的概率
        \[
        P_i \propto p(\text{单元 }i)\cdot\exp(-\beta\cdot\mathrm{recall}_i),
        \]
        其中 $p$ 是其持久性,$\mathrm{recall}_i$ 是近期被回放的次数,$\beta$ 是控制多样性的超参数。
        这使高持久性且近期较少被复习的记忆项有更高优先级。

  \item 基于结构的回放增强:在回放一个子图记忆项时,并非孤立地输入该子图,而是将其“植入”
        到当前快照 $G_t$ 的一个相似上下文中(例如,找到当前图中一个与记忆子图表征相似的节点簇,
        将记忆子图作为附加结构注入计算图),以模拟该核心结构在新环境下的作用,增强知识的可迁移性。
\end{enumerate}
\subsection{策略二:拓扑正则化}

正则化方法通过约束参数更新来保护旧知识。我们提出 Topology-Aware Elastic Weight Consolidation (TA-EWC),将GED分析得出的拓扑稳定性信息融入参数重要性估计中,实现正则化的“拓扑感知”。

\begin{enumerate}[label=\arabic*.]

  \item 计算拓扑加权的 Fisher 信息矩阵:  
        在标准的 EWC 中,参数 $\theta_i$ 的重要性正比于 Fisher 信息矩阵对角线元素 $F_i$,它度量了该参数对预测结果的敏感度。我们引入拓扑权重进行修正。
        \begin{itemize}[leftmargin=*,nosep]
          \item 首先,在任务 $A$(对应历史图数据)上,对于每个训练样本(节点),计算其损失对参数的梯度
                $\nabla_{\!\theta}\mathcal{L}_A\!\bigl(f_\theta(v)\bigr)$。

          \item 然后,计算拓扑加权 Fisher 信息:
                \[
                \widetilde{F}^{\,(A)}_i =
                \frac{1}{|\mathcal{V}_A|}
                \sum_{v\in\mathcal{V}_A}
                \underbrace{\bigl(\alpha(v)+\varepsilon\bigr)^{-1}}_{\text{稳定性权重}}
                \bigl[\nabla_{\!\theta}\mathcal{L}_A\!\bigl(f_\theta(v)\bigr)\bigr]^2_i,
                \]
                这里 $\alpha(v)$ 是顶点活跃度。关键洞察:对稳定顶点(低 $\alpha(v)$)预测正确的参数更为重要,因为这些顶点表征了网络的静态核心。因此,我们用活跃度的倒数进行加权,赋予稳定点对应的梯度平方更高权重。
        \end{itemize}

  \item 定义正则化损失:  
        当学习新任务 $B$(对应新快照)时,总损失函数为
        \[
        \mathcal{L}_{\text{total}}(\theta)
        = \mathcal{L}_B(\theta)
        + \frac{\lambda}{2}\sum_i \widetilde{F}^{\,(A)}_i(\theta_i - \theta^*_i)^2,
        \]
        其中 $\theta^*$ 是任务 $A$ 训练结束后的最优参数,$\lambda$ 控制正则化强度。
\end{enumerate}

相比于其他的算法,TA-EWC的主要优势是正则化更具有解释性,与图结构紧密耦合,而且可与回放策略自然融合。

\chapter{算法实现与复杂性分析}
鉴于GED精确计算的NP-hard特性,我们首先讨论高效的近似计算优化方法,随后描述整体算法流程,并分析其时间与空间复杂度,论证该框架在实际大规模动态图场景中的可行性。
\section{近似计算优化}
精确计算GED对于大规模图是不可行的。因此,我们采用并改进一种经典的近似算法——基于二分图匹配的GED计算(Bipartite Graph Matching for GED, BGM-GED),其核心思想是将复杂的图匹配问题分解为顶点匹配与局部结构评估相结合的过程。

该算法输入两个标记图$G_1=(V_1,E_1,l_(V_1),l_(E_1))$,$G_2=(V_2,E_2,l_(V_2),l_(E_2))$和代价函数,可以得到近似的图编辑距离和近似的顶点映射关系。
\section{算法流程描述}
\subsubsection*{BGM-GED 算法步骤}

\begin{enumerate}[label=\arabic*.]

\item 构造代价矩阵 $\mathbf{C}$:
\begin{itemize}[leftmargin=*,nosep]
  \item 矩阵维度为 $(|V_1|+1)\times(|V_2|+1)$,额外的行和列代表顶点删除和插入。
  \item 对于 $i\le |V_1|,j\le |V_2|$,计算将 $u_i\in V_1$ 替换为 $u_j\in V_2$ 的代价 $C[i][j]$。它包括两部分:
  \begin{itemize}
    \item[a)] \textbf{顶点替换代价:} $C\!\bigl(l_{V_1}(u_i),l_{V_2}(u_j)\bigr)$。
    \item[b)] \textbf{局部邻接结构差异的近似代价:} 该代价估计由该映射引发的边结构变化。我们采用一种高效启发式方法:
    \begin{itemize}
      \item 令 $N_i$ 为 $u_i$ 在 $G_1$ 中的邻接边集合,$N_j$ 为 $u_j$ 在 $G_2$ 中的邻接边集合。
      \item 计算两个顶点度数的绝对差乘以单位边操作代价作为简单估计:
            $|\deg(u_i)-\deg(u_j)|\cdot\min(C_{ie},C_{de})$。
      \item (可选但更精确)对 $N_i$ 和 $N_j$ 中的边标签进行直方图统计,计算直方图间的 Earth Mover's Distance 或 $L_1$ 距离,再乘以边操作代价。
    \end{itemize}
  \end{itemize}
  \item 对于 $i\le |V_1|,j=|V_2|+1$,$C[i][j]=C_{du}$(删除顶点 $u_i$)。
  \item 对于 $i=|V_1|+1,j\le |V_2|$,$C[i][j]=C_{iu}$(插入顶点 $u_j$)。
\end{itemize}

\item 求解最小代价分配:\\
使用匈牙利算法(Hungarian Algorithm)或更快的 Jonker-Volgenant 算法求解上述代价矩阵的最小代价分配,得到最优的顶点映射 $\Phi$。该映射确定了哪些顶点被替换、删除或插入。

\item 推导边编辑操作与计算总代价:
\begin{itemize}[leftmargin=*,nosep]
  \item 根据顶点映射 $\Phi$,确定边编辑操作:
  \begin{itemize}
    \item 对于每条边 $e_1=(u_i,u_{i'})\in E_1$,如果 $\Phi(u_i)$ 和 $\Phi(u_{i'})$ 均不是 $\varepsilon$ 且边 $e_2=\bigl(\Phi(u_i),\Phi(u_{i'})\bigr)\in E_2$,则考虑边替换(若边标签不同,代价为 $C_e\!\bigl(l_E(e_1),l_E(e_2)\bigr)$,否则为 $0$)。
    \item 如果 $\Phi(u_i)$ 和 $\Phi(u_{i'})$ 均不是 $\varepsilon$ 但 $e_2\notin E_2$,则需在 $G_2$ 中删除此边,代价为 $C_{de}$。
    \item 对于每条边 $e_2=(u_j,u_{j'})\in E_2$,如果其端点不是由 $\Phi$ 从 $G_1$ 映射而来(即没有 $u_i$ 使得 $\Phi(u_i)=u_j$ 或没有 $u_{i'}$ 使得 $\Phi(u_{i'})=u_{j'}$),则需在 $G_1$ 中插入此边,代价为 $C_{ie}$。
  \end{itemize}
  \item 总近似 GED 为所有顶点和边操作代价之和。
\end{itemize}
\end{enumerate}

\subsubsection*{针对动态图序列的优化(增量 GED 计算)}
在动态图序列中,相邻快照 $G_{t-1}$ 和 $G_t$ 通常高度相似。我们可以利用这一特性显著加速计算:
\begin{itemize}[leftmargin=*,nosep]
  \item \textbf{热启动映射:} 使用上一轮计算得到的 $G_{t-2}\to G_{t-1}$ 的顶点映射 $\Phi_{t-2,t-1}$ 作为初始解,通过局部调整(如仅对变化区域内的顶点重新匹配)来求解 $\Phi_{t-1,t}$。
  \item \textbf{局部搜索:} 当变化很小时(例如,仅少量顶点/边增删),可以限制匈牙利算法的搜索空间,仅考虑对可能发生变化的顶点进行重新匹配。
\end{itemize}

\subsubsection*{复杂性分析(单次 BGM-GED)}
\begin{itemize}[leftmargin=*,nosep]
  \item \textbf{时间复杂性:} 步骤 1 中构造代价矩阵需要 $O(|V_1|\cdot|V_2|\cdot d)$ 时间,其中 $d$ 是平均计算局部邻接结构差异的代价(对于简单的度数差异法,$d=O(1)$)。步骤 2 中匈牙利算法的时间复杂度为 $O\!\bigl(\max(|V_1|,|V_2|)^3\bigr)$。步骤 3 需要 $O(|E_1|+|E_2|)$ 时间。因此,总时间复杂度为 $O(n^3+m)$,其中 $n=\max(|V_1|,|V_2|)$, $m=\max(|E_1|,|E_2|)$。
  \item \textbf{空间复杂性:} 主要存储代价矩阵,需要 $O(|V_1|\cdot|V_2|)$ 空间。
\end{itemize}
\chapter{总结与展望}
\section{结论}
本文深入探讨了离散数学,特别是图编辑距离,在图持续学习这一前沿领域中的应用。面对动态图序列中灾难性遗忘的核心挑战,我们认识到必须超越简单地应用通用持续学习技术,而需开发与图结构演化语义深度耦合的方法。

本研究的主要贡献可总结为以下三点:

\begin{enumerate}[label=\arabic*.]
  \item \textbf{理论桥梁的构建:}  
  我们系统性地梳理了从静态图论概念(子图、同构)到动态图相似性度量(图编辑距离)的逻辑脉络,强调了 GED 在刻画“近似同构”和量化渐进演化方面的独特价值,为动态图分析建立了从定性到定量的理论桥梁。

  \item \textbf{动态建模框架的创新:}  
  提出了一个基于 GED 的分层动态建模框架。该框架首先通过构建演化图和计算结构单元持久性/活跃度,将宏观的序列差异映射为微观的、具有明确拓扑语义的演化属性标签,实现了对图结构变化的精细化度量。

  \item \textbf{持续学习策略的设计:}  
  基于上述量化模型,我们创新性地提出了两种策略:
  \begin{itemize}
    \item \textbf{基于 GED 的经验回放(GGR):} 通过选择性回放演化过程中持久性高的核心子图与稳定顶点,使模型复习最应被长期记忆的拓扑知识。
    \item \textbf{拓扑感知的弹性权重巩固(TA-EWC):} 将顶点活跃度融入参数重要性估计中,从而对稳定结构对应的模型参数施加更强的保护性约束。
  \end{itemize}
  这两种策略均利用 GED 解析出的演化信息,实现了“理解变化并据此行动”的智能化持续学习。
\end{enumerate}

此外,我们针对 GED 的计算复杂性,详细阐述了基于二分图匹配的近似算法(BGM-GED)及其在动态场景下的增量优化,并从时空复杂度角度论证了整体框架的可行性。本文工作表明,将离散数学的严谨理论与机器学习的前沿需求相结合,能够催生具有更强可解释性和性能潜力的新方法。
\section{未来工作}
尽管本研究提出了一个系统的框架,但仍有诸多方向值得在未来进一步探索:

\begin{enumerate}[label=\arabic*.]
  \item \textbf{GED 计算的进一步加速:}  
  BGM-GED 算法复杂度仍为 $O(n^3)$。未来可探索基于图神经网络(GNN)的端到端 GED 预测模型,或将大规模图匹配问题分解为基于社区结构的层次化匹配,以实现近乎线性的近似计算效率。

  \item \textbf{更复杂的动态场景建模:}  
  本文主要处理离散时间快照序列。未来可将框架扩展至连续时间动态图,研究如何定义连续时间上的“瞬时编辑距离”,并设计相应的在线学习与回放机制。

  \item \textbf{策略的融合与自动化:}  
  GGR 与 TA-EWC 可以进一步融合,并探索自适应机制,根据演化图的模式(平稳期或剧变期)动态调整回放比例、正则化强度等超参数,实现完全自适应的图持续学习。

  \item \textbf{更广泛的应用验证:}  
  在更多样化、更大规模的动态图数据集(如不断演化的全球贸易网络、动态知识图谱)上验证本框架的通用性与鲁棒性,并探索其在时序预测、异常检测等下游任务中的表现。

  \item \textbf{理论分析的深化:}  
  从学习理论角度,分析基于拓扑持久性的回放策略对模型泛化误差界的影响,为方法提供更坚实的理论保证。
\end{enumerate}

总之,图持续学习是一个充满活力的交叉研究领域。我们相信,持续深化离散数学与机器学习的对话,将为理解和处理日益复杂的动态世界提供更强大的工具。